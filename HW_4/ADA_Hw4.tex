\documentclass{homework}
\author{許崴棠}
\title{ADA Homework 4}

\begin{document}
\maketitle
\section*{Problem 0}
\begin{itemize}
    \item \textbf{Problem 1:} 
    \item \textbf{Problem 2:} 
    \item \textbf{Problem 3:} 
\end{itemize}
\clearpage
\section*{Problem 1}
\begin{itemize}
    \item[(a)]
    To determine if a 2-CNF problem is in 2-SAT, we can regard the 2-CNF formula as a set of clause, take $\phi_1$ as an example we can regard $\phi_1$ as 
    \[\phi_1 = \{(a_1\vee \lnot a_2), (a_3 \vee a_1), (\lnot a_1 \vee \lnot a_1)\}\] 
    and by De Morgan law, we can rewrite $\phi_1$ as 
    \[\phi_1 ' = \{\lnot(\lnot a_1 \wedge  a_2), \lnot(\lnot a_3 \wedge \lnot a_1), \lnot(a_1 \wedge a_1)\}\]
    Consider a directed graph $G$ where $V(G) = \{a_1, a_2,\ldots,a_n, \lnot a_1, \lnot a_2,\ldots,\lnot a_n\}$. And to make $\lnot(x \wedge y)$ in $\phi_1'$ true, we can add the edge of $x \rightarrow \lnot y$ and $y \rightarrow \lnot x$ since that if $x$ is right, $\lnot y$ must be right, if  $y$ is right, $\lnot x$ must be right.
    After then, we can find the SCCs by the kosaraju's algorithm, and we can check if a 2-CNF formula is solvable simply by checking if there exist a $a_i$ where $a_i$ and $\lnot a_i$ are in the same SCC, since if $x$ is able to reach $y$, it means that ``if $x$ is true, then $y$ must be true '', and if $a_i$ and $\lnot a_i$ are in the same SCC it means that the statement $a_i$ will be true if and only if statement $\lnot a_i$ is true, which is obviously wrong. Thus we can know that there will be at least one set of $a$ that can solve the answer, only if there is no $a_i$ and $\lnot a_i$ in the same scc.

    \item[(b)] 
    \begin{itemize}
        \item[(b-1)]%! not done
        % for every clause, how close is their clause(i)
        % 3-SAT can go to this(a1,a2,a3)>(-a1,a2)&(-a2,a3)&(-a3,a1)&(-a1,a2)&(-a2,a3)&(-a3,a1)
        % for all x(dist) in 2 sat, there can be 0.001x(dist) in 3 sat 
        % 

        \item[(b-2)]%! not done
        

    \end{itemize}
    \item[(c)] 
    \begin{itemize}
        \item[(c-1)]%! not done
        

        \item[(c-2)]%! not done
        

    \end{itemize}
\end{itemize}
\clearpage

\section*{Problem 2}
\begin{itemize}
    \item[(a)]
    By observe, we can tell that after we put $\alpha$ items into the current box, we'll had to spend $2\alpha$ time to make a new box, so when we're putting new items into the box, we can save 6 dollars to the bank per item, therefore, after we put $\alpha$ items into the box with the size $2\alpha$, we can directly spend the $4\alpha$ dollars we've saved in the bank to make a new box with the size of $4\alpha$, and spend $2\alpha$ dollars to move the $2\alpha$ items in the original box. By doing this, we can say that the time complexity of putting n items into the box are $O(6n) = O(n)$. 
    
    \item[(b)]
    It is impossible to amortize the time complexity to $O(n)$ because that when $n = s + 1$, we need to spend $s^2$ time to make a new box, and it would be impossible to amortize a $s^2$ step to $s$ step to make the overall time complexity $O(n)$.

    \item[(c)]
    By observe, we can tell that after we put $\alpha$ items into the current box, we'll had to spend $\alpha^2$ time to make a new box. When the box is full when we put the $i_{th}$ item into the box, we had to make sure the previous $i-\sqrt[]{i}$ items can afford the price of making new box and move the every items into the new box. So when we're moving $n^2$ items to a new box, we had to pay ${(n^2)}^2+n^2$ dollars, and it should be pay by the ${n+1}_{th}$ to ${n^2}_{th}$ items. By math, we can know that to put the if we assume every $i_{th}$ item had to pay $\alpha i +\beta$ dollars to bank. Since we already knew the first box (size 2) can be enlarge to the second box if every one put $2n$ dollars into the bank.\\
    We had to calculate the general case, which is 
    \[{(n^2)}^2+n^2 \leq \alpha \frac{(n^2+n+1)*(n^2-n)}{2} + \beta (n^2-n)\;\forall\; n \geq 2\]
    where $n^2$ is the current box size.\\
    By math, we can tell that the tightest $\alpha$ is 2 and the tightest $\beta$ is 3. Therefore, if every $i_{th}$ insert put $2i+3$ dollars into the bank, we can safely make a $n^2$ box and move the previous $n_{th}$ items into the new box only by using the ${n}^2+n$ the previous ${\sqrt[]{n}+1}_{th}$ to $n_{th}$ items had saved.

    \item[(d)]
    First, we assume the worst case, which happens when the manager come and check after every time Nathan put an item into the box.\\
    In the cost section of the below table, it represent the (cost of insert + cost of move)\\
    \begin{table}[H]
        \centering
        \begin{tabular}{|c|c|c|c|c|c|}
            \hline
            i  & cost & i  & cost & i  & cost \\\hline
            1  & 1+0  & 6  & 1+1  & 11 & 1+1  \\\hline
            2  & 1+1  & 7  & 1+1  & 12 & 1+1  \\\hline
            3  & 1+3  & 8  & 1+1  & 13 & 1+1  \\\hline
            4  & 1+1  & 9  & 1+9  & 14 & 1+1  \\\hline
            5  & 1+5  & 10 & 1+1  & 15 & 1+1  \\\hline
        \end{tabular}
    \end{table}
    By the table above, we can find out that the maximum averagetime complexity happens when we put $n=2^k+1$ item into the boxes, which can take at most $2n-1+2(2^k-1)=4n-3$ times, and we can tell the time complexity is $O(4n-3)=O(n)$.

    \item[(e)] %! not done
    

\end{itemize}
\clearpage


\end{document}